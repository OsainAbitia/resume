%----------------------------------------------------------------------------------------
%	DOCUMENT DEFINITION
%----------------------------------------------------------------------------------------

% we use article class because we want to fully customize the page
\documentclass[10pt,A4]{article}	


%----------------------------------------------------------------------------------------
%	ENCODING
%----------------------------------------------------------------------------------------

%we use utf8 since we want to build from any machine
\usepackage[utf8]{inputenc}	

%----------------------------------------------------------------------------------------
%	LOGIC
%----------------------------------------------------------------------------------------

\usepackage{xifthen}
\usepackage{calc}
\usepackage[hidelinks]{hyperref}

%----------------------------------------------------------------------------------------
%	FONT
%----------------------------------------------------------------------------------------

% some tex-live fonts - choose your own

%\usepackage[defaultsans]{droidsans}
%\usepackage[default]{comfortaa}
%\usepackage{cmbright}
\usepackage[default]{raleway}
%\usepackage{fetamont}
%\usepackage[default]{gillius}
%\usepackage[light,math]{iwona}
%\usepackage[thin]{roboto} 

% set font default
\renewcommand*\familydefault{\sfdefault} 	
\usepackage[T1]{fontenc}

% more font size definitions
\usepackage{moresize}		

% font icons package
\usepackage{fontawesome}

%----------------------------------------------------------------------------------------
%	PAGE LAYOUT  DEFINITIONS
%----------------------------------------------------------------------------------------

%define page styles using geometry
\usepackage[a4paper]{geometry}		

% for example, change the margins to 2 inches all round
\geometry{top=2cm, bottom=2cm, left=2cm, right=2cm} 	

% use customized header
\usepackage{fancyhdr}				
\pagestyle{fancy}

%less space between header and content
\setlength{\headheight}{-5pt}		

% customize header entries
\lhead{}
\rhead{}
\chead{}

%indentation is zero
\setlength{\parindent}{0mm}

%----------------------------------------------------------------------------------------
%	GRAPHICS DEFINITIONS
%---------------------------------------------------------------------------------------- 

% for drawing graphics and charts
\usepackage{tikz}
\usetikzlibrary{shapes, backgrounds}

% use to vertically center content
% credits to: http://tex.stackexchange.com/questions/7219/how-to-vertically-center-two-images-next-to-each-other
\newcommand{\vcenteredinclude}[1]{\begingroup
	\setbox0=\hbox{\includegraphics{#1}}%
	\parbox{\wd0}{\box0}\endgroup}

% use to vertically center content
% credits to: http://tex.stackexchange.com/questions/7219/how-to-vertically-center-two-images-next-to-each-other
\newcommand*{\vcenteredhbox}[1]{\begingroup
	\setbox0=\hbox{#1}\parbox{\wd0}{\box0}\endgroup}

%----------------------------------------------------------------------------------------
%	ICON-SET EMBEDDING
%---------------------------------------------------------------------------------------- 

% at this point we simplify our icon-embedding by simply referring to a set of png images.
% if you find a good way of including svg without conflicting with other packages you can
% replace this part
\newcommand{\icon}[2]{\colorbox{black}{\makebox(#2, #2){\textcolor{white}{\large\csname fa#1\endcsname}}}}	%icon shortcut
\newcommand{\icontext}[3]{ 						%icon with text shortcut
	\vcenteredhbox{\icon{#1}{#2}}\hspace{0.2cm}\vcenteredhbox{\textcolor{black}{#3}}
}

%----------------------------------------------------------------------------------------
% 	HEADER
%----------------------------------------------------------------------------------------

% remove top header line
\renewcommand{\headrulewidth}{0pt} 

%remove botttom header line
\renewcommand{\footrulewidth}{0pt}	  	

%remove pagenum
\renewcommand{\thepage}{}	

%remove section num		
\renewcommand{\thesection}{}			


%----------------------------------------------------------------------------------------
%
% 	TIKZ GRAPHICS
%
%----------------------------------------------------------------------------------------

\newcounter{a}
\newcounter{b}
\newcounter{c}
\newcounter{barcount}

%----------------------------------------------------------------------------------------
% 	BAR CHART
%----------------------------------------------------------------------------------------

% draw a bar chart
% param 1: width
% param 2: height
% param 3: border color
% param 4: label text color
% param 5: label bg color
% param 6: cat 1 color
\newenvironment{barchart}[8]{
	
	\newcommand{\barwidth}{0.35}
	\newcommand{\barsep}{0.2}
	
	% param 1: overall percent
	% param 2: label
	% param 3: cat 1 percent
	% param 4: cat 2 percent
	% param 5: cat 3 percent
	\newcommand{\baritem}[5]{
		
		\pgfmathparse{##3+##4+##5}
		\let\perc\pgfmathresult
		
		\pgfmathparse{#2}
		\let\barsize\pgfmathresult
		
		\pgfmathparse{\barsize*##3/100}
		\let\barone\pgfmathresult
		
		\pgfmathparse{\barsize*##4/100}
		\let\bartwo\pgfmathresult
		
		\pgfmathparse{\barsize*##5/100}
		\let\barthree\pgfmathresult
		
		\pgfmathparse{(\barwidth*\thebarcount)+(\barsep*\thebarcount)}
		\let\barx\pgfmathresult
		
		\filldraw[fill=#6, draw=none] (0,-\barx) rectangle (\barone,-\barx-\barwidth);
		\filldraw[fill=#7, draw=none] (\barone, -\barx) rectangle (\barone+\bartwo,-\barx-\barwidth);
		\filldraw[fill=#8, draw=none] (\barone+\bartwo,-\barx ) rectangle (\barone+\bartwo+\barthree,-\barx-\barwidth);
		
		\node [label=180:\colorbox{#5}{\textcolor{#4}{##2}}] at (0,-\barx-0.175) {};
		\addtocounter{barcount}{1}
	}
	\begin{tikzpicture}
	\setcounter{barcount}{0}
	
}
{\end{tikzpicture}}

%----------------------------------------------------------------------------------------
% 	BUBBLE CHART
%----------------------------------------------------------------------------------------
\newcommand{\bubble}[5]{
	\definecolor{tmpcol}{RGB}{50,50,#5}
	% slice
	\filldraw[fill=black,draw=none] (#1,0.5) circle (#3);
	
	% outer label
	\node[label=\textcolor{black}{#4}] at (#1,0.7) {};
}

\newcommand{\bubbles}[2]{
	%reset counters
	\setcounter{a}{0}
	\setcounter{c}{150}
	\begin{tikzpicture}[scale=3]
	\foreach \p/\t in {#1} {
		\addtocounter{a}{1}
		\bubble{\thea/2}{\theb}{\p/25}{\t}{1\p0}
	}
	\end{tikzpicture}
}


%----------------------------------------------------------------------------------------
%	custom sections
%----------------------------------------------------------------------------------------

% create a coloured box with arrow and title as cv section headline
% param 1: section title
%
\newcommand{\cvsection}[1] {
	\textcolor{white}{\MakeUppercase{\textbf{#1}}}
}

\newcommand{\cvsect}[1]{
	\colorbox{black}{{\cvsection{#1}}}\\\\%
}

%----------------------------------------------------------------------------------------
% CUSTOM LOREM IPSUM
%----------------------------------------------------------------------------------------
\newcommand{\lorem}{Lorem ipsum dolor sit amet, consectetur adipiscing elit. Donec a diam lectus.}

%----------------------------------------------------------------------------------------
% ENTRY LIST
%----------------------------------------------------------------------------------------
\usepackage{tabularx}

\setlength{\tabcolsep}{0pt}
\newenvironment{entrylist}{%
	\begin{tabular*}{\textwidth}[t]{@{\extracolsep{\fill}}ll}
	}{%
	\end{tabular*}
}

\newcommand{\entry}[4]{%
	\parbox[t]{3.5cm}{%
		#1%
	}%
	&\parbox[t]{14cm}{%
		\textbf{#2}%
		\hfill%
		{\footnotesize \textbf{\textcolor{black}{#3}}}\\%
		#4%
	}\\\\}

\newcommand{\slashsep}{
	\hspace{2mm}/\hspace{2mm}
}

%----------------------------------------------------------------------------------------
%	DOCUMENT CONTENT
%----------------------------------------------------------------------------------------
\begin{document}
	
	%----------------------------------------------------------------------------------------
	%	TITLE HEADLINE
	%----------------------------------------------------------------------------------------
	\begin{minipage}[t]{0.45\textwidth}\hrule height 0pt width 0pt%
		\colorbox{black}{{\HUGE\textcolor{white}{\textbf{\MakeUppercase{Osain Abitia}}}}}%
		
		\vspace{1mm}\LARGE{Software Engineer}
	\end{minipage}%
	\begin{minipage}[t]{0.3\textwidth}\hrule height 0pt width 0pt%
		\small%
		\icontext{MapMarker}{12}{Durango, Mexico}\\
		\icontext{At}{12}{\href{mailto:osain.abitia@gmail.com}{osain.abitia@gmail.com}}\\	
	\end{minipage}%
	\begin{minipage}[t]{0.3\textwidth}\hrule height 0pt width 0pt%
		\small%
		\icontext{Github}{12}{\href{https://github.com/OsainAbitia}{github.com/OsainAbitia}}\\
	\end{minipage}%
	
	% manage space by reducing font size
	\small%
	\vspace{1cm}
	
	%----------------------------------------------------------------------------------------
	%	SKILLS AND TECHNOLOGIES
	%----------------------------------------------------------------------------------------
	
	\cvsect{Who Am I?}%
	\begin{minipage}[t]{1\textwidth}%
		I'm a certified \textbf{AWS Solutions Architect and Software Engineer} with a passion for 
		leading teams to build impactful products that address real-world needs. With a
		strong background in managing cloud technologies like \textbf{AWS}, and a strong background
		with \textbf{programming languages like Python}, I have solid experience in designing and building secure,
		reliable, and scalable systems within the cloud.
	\end{minipage}%
	\hfill
	\vspace{0.5cm}
	
	%----------------------------------------------------------------------------------------
	%	Certificates & Languages
	%----------------------------------------------------------------------------------------
	
	
	\begin{minipage}[t]{0.8\textwidth}\hrule height 0pt width 0pt%
		\cvsect{Certificates}
		\href{https://www.credly.com/badges/dbb4510d-c80e-4bdf-bf9b-52e370013a9a}{\underline{\textbf{AWS Certified Solutions Architect (Associate)}} - issued by Amazon Web Services}\hfil\break
		\href{https://www.credly.com/badges/4f129c39-edb2-410c-91aa-d8bf8617706f}{\underline{\textbf{AWS Certified Cloud Practitioner}} - issued by Amazon Web Services}\hfil\break
		\href{https://www.credly.com/badges/1e0ac89f-4e38-4e91-93aa-3a62e7c33709}{\underline{\textbf{AWS Well-Architected Proficient}} - issued by Amazon Web Services}\hfil\break
		\href{https://www.credly.com/badges/4f7bb440-1f63-4b79-a7b1-116f16da0656}{\underline{\textbf{AWS Partner: Accreditation (Technical)}} - issued by Amazon Web Services}\hfil\break
		\href{https://www.credly.com/badges/c35e5c25-1938-4810-9489-1871644d7175}{\underline{\textbf{AWS Partner: Accreditation (Business)}} - issued by Amazon Web Services}\hfil\break
	\end{minipage}%
	\hspace{1cm}
	\begin{minipage}[t]{0.3\textwidth}\hrule height 0pt width 0pt%
		\cvsect{Languages}
		\textbf{Spanish} - Native\\
		\textbf{English} - \href{https://cert.efset.org/hqaFfD}{C1}\\
	\end{minipage}%
	\hfill
	\vspace{0.5cm}

	%----------------------------------------------------------------------------------------
	%	EXPERIENCE
	%----------------------------------------------------------------------------------------
	
	\cvsect{Experience}
	\begin{entrylist}
		\entry
		{Mar 2022 – Present}
		{Senior Software Engineer 2 - Platform Team}
		{Apli Jobs} % Mention Infrastructure standardization, Dockerize services, configuring API Gateways, 
		{In this role, I led process standardization for infrastructure creation using Terraform modules,
		\underline{reducing the development cycle by 80\%}. This efficiency gain allowed me to take the initiative to
		start Dockerizing services and deploy them on ECS, significantly streamlining deployment processes.\\
		\hfil\break
		I also \underline{designed and exposed REST APIs through API Gateway}, enhancing system integration. To ensure
		system reliability under load, I introduced \underline{Stress Testing with Artillery}, creating complex API
		workflows to simulate real user interactions, which identified bottlenecks and accelerated service
		release times.\\
			\texttt{AWS}\slashsep\texttt{Python}\slashsep\texttt{Docker}\slashsep\texttt{GitHub}\slashsep\texttt{Artillery Load Tests}\slashsep\texttt{Terraform}\slashsep\texttt{Mentorship}}
		\entry
		{Jan 2022 – Mar 2022}
		{Software Engineer - Cloud Team}
		{Apli Jobs}
		{As the first and only Cloud Engineer during my initial 6 months at Apli, I spearheaded the design
		and \underline{implementation of cloud-based architecture for cross-team projects}, collaborating with Data Science,
		Product, and Engineering teams. Utilizing AWS services such as EMR, ECS, DynamoDB, Lambda, AWS Backup,
		RDS, and Redis, I also successfully implemented cost optimization mechanisms that \underline{reduced the company's costs in around 10\% for the first year}.\\
		\hfil\break
		I initiated and led internal efforts to bolster company culture and enforce security protocols.
		I was a \underline{key member of the committee that introduced the on-call program} to the Engineering team, which
		resulted in a response time against incident of less than 4 hours. I also led the ISO audit and
		certification process for cloud infrastructure controls, ensuring our architecture met the highest standards of security, 
		scalability, and reliability, which culminated in achieving ISO 27k1.\\
		\hfil\break
		As the team expanded, \underline{I took on a mentor and leadership role}, establishing processes that improved
		code quality, including enhanced documentation, automated security checks in CI/CD, and regular
		knowledge-sharing meetings.\\
			\texttt{AWS}\slashsep\texttt{Python}\slashsep\texttt{Docker}\slashsep\texttt{GitHub}\slashsep\texttt{Terraform}\slashsep\texttt{GCP}}
		\entry
		{Feb 2021 – Jan 2022}
		{Backend Developer + AWS Cloud Engineer}
		{Galaxy Technology}
		{As a Backend Engineer and Team Lead, I crafted REST APIs and oversaw database
		management using Node.js with Express and Sequelize. My role extended to hosting
		essential services within AWS, employing tools like Lambda, API Gateway,
		AppSync, DynamoDB, SNS, and SQS. Collaborating closely with customers, I addressed
		change requests and implemented new functionalities, enhancing our services to
		meet evolving needs.\\
			\texttt{Node.js}\slashsep\texttt{Express}\slashsep\texttt{Sequelize}\slashsep\texttt{AWS}\slashsep\texttt{GraphQL}\slashsep\texttt{Python}\slashsep\texttt{Serverless}}
	\end{entrylist}
	\\\\

	%----------------------------------------------------------------------------------------
	%	Education
	%----------------------------------------------------------------------------------------

	\cvsect{Education and Personal Projects}
	\begin{entrylist}
		\entry
		{2022}
		{NFarm}
		{Zencon International Hackathon}
		{Developed a basic MVP for a blockchain-based digital traceability system for cattle using NFTs
		and smart contracts, enabling farmers to ensure product quality and reduce food waste. Secured
		first place in the NFTs category.}
		\entry
		{2019}
		{Lung Cancer Detection System with AI - Damai}
		{Expo Science International}
		{Developed an MVP system using Convolutional Neural Networks and image processing to detect lung
		cancer in its early stages.Selected as one of the top 100 projects representing Mexico at Expo
		Science International in Abu Dhabi, UAE.}
		\entry
		{2017 – 2020}
		{Software Engineering}
		{Universidad Politécnica de Durango (UNIPOLI)}
		{Graduated with a 9.42 score.}
	\end{entrylist}%
	\\\\

	%----------------------------------------------------------------------------------------
	%	Other
	%----------------------------------------------------------------------------------------

	\cvsect{Other}
	\begin{entrylist}
		\entry
		{2022}
		{How to pitch your project idea?}
		{Durango, Mexico}
		{Participated in a workshop for students at the \textit{Universidad Politécnica de Durango} on the key points
		to consider when pitching an idea to potential investors or during a hackathon, explaining the
		sections that an elevator pitch should have to create a big impact, using as an example my experience
		at the Zencon International Hackathon and Expo Science International.}
		\entry
		{2020}
		{Volunteer at World Sustainable Development Forum}
		{Durango, Mexico}
		{Translator for the speakers at the World Sustainable Development Forum, translating from English to Spanish
		and vice versa during the 3 days the event lasted. This forum had as its main objective to discuss climate
		change to encourage the citizens to take action against it.}
	\end{entrylist}%
	\\\\

	%----------------------------------------------------------------------------------------
	%	Hobbies
	%----------------------------------------------------------------------------------------
	
	
	\begin{minipage}[t]{0.5\textwidth}\hrule height 0pt width 0pt%
		\cvsect{Hobbies}
		{\textbf{Continuous Learning:} I regularly read technical books and take courses on new technologies and tools to enhance my skills.}\\
		{\textbf{Work-Life Balance:} I prioritize balancing work with personal life to spend quality time with my family.}\\
		{\textbf{Podcasts:} I'm a fan of paranormal podcasts. Feel free to share any recommendations with me.}\\
		\hfil\break
		{For book, course, or podcast recommendations, you can reach me at \href{mailto:osain.abitia@gmail.com}{\textbf{osain.abitia@gmail.com}}.}\\

	\end{minipage}%
	
\end{document}
