%----------------------------------------------------------------------------------------
%	DOCUMENT DEFINITION
%----------------------------------------------------------------------------------------

% we use article class because we want to fully customize the page
\documentclass[10pt,A4]{article}	


%----------------------------------------------------------------------------------------
%	ENCODING
%----------------------------------------------------------------------------------------

%we use utf8 since we want to build from any machine
\usepackage[utf8]{inputenc}	

%----------------------------------------------------------------------------------------
%	LOGIC
%----------------------------------------------------------------------------------------

\usepackage{xifthen}
\usepackage{calc}
\usepackage[hidelinks]{hyperref}

%----------------------------------------------------------------------------------------
%	FONT
%----------------------------------------------------------------------------------------

% some tex-live fonts - choose your own

%\usepackage[defaultsans]{droidsans}
%\usepackage[default]{comfortaa}
%\usepackage{cmbright}
\usepackage[default]{raleway}
%\usepackage{fetamont}
%\usepackage[default]{gillius}
%\usepackage[light,math]{iwona}
%\usepackage[thin]{roboto} 

% set font default
\renewcommand*\familydefault{\sfdefault} 	
\usepackage[T1]{fontenc}

% more font size definitions
\usepackage{moresize}		

% font icons package
\usepackage{fontawesome}

%----------------------------------------------------------------------------------------
%	PAGE LAYOUT  DEFINITIONS
%----------------------------------------------------------------------------------------

%define page styles using geometry
\usepackage[a4paper]{geometry}		

% for example, change the margins to 2 inches all round
\geometry{top=2cm, bottom=2cm, left=2cm, right=2cm} 	

% use customized header
\usepackage{fancyhdr}				
\pagestyle{fancy}

%less space between header and content
\setlength{\headheight}{-5pt}		

% customize header entries
\lhead{}
\rhead{}
\chead{}

%indentation is zero
\setlength{\parindent}{0mm}

%----------------------------------------------------------------------------------------
%	GRAPHICS DEFINITIONS
%---------------------------------------------------------------------------------------- 

% for drawing graphics and charts
\usepackage{tikz}
\usetikzlibrary{shapes, backgrounds}

% use to vertically center content
% credits to: http://tex.stackexchange.com/questions/7219/how-to-vertically-center-two-images-next-to-each-other
\newcommand{\vcenteredinclude}[1]{\begingroup
	\setbox0=\hbox{\includegraphics{#1}}%
	\parbox{\wd0}{\box0}\endgroup}

% use to vertically center content
% credits to: http://tex.stackexchange.com/questions/7219/how-to-vertically-center-two-images-next-to-each-other
\newcommand*{\vcenteredhbox}[1]{\begingroup
	\setbox0=\hbox{#1}\parbox{\wd0}{\box0}\endgroup}

%----------------------------------------------------------------------------------------
%	ICON-SET EMBEDDING
%---------------------------------------------------------------------------------------- 

% at this point we simplify our icon-embedding by simply referring to a set of png images.
% if you find a good way of including svg without conflicting with other packages you can
% replace this part
\newcommand{\icon}[2]{\colorbox{black}{\makebox(#2, #2){\textcolor{white}{\large\csname fa#1\endcsname}}}}	%icon shortcut
\newcommand{\icontext}[3]{ 						%icon with text shortcut
	\vcenteredhbox{\icon{#1}{#2}}\hspace{0.2cm}\vcenteredhbox{\textcolor{black}{#3}}
}

%----------------------------------------------------------------------------------------
% 	HEADER
%----------------------------------------------------------------------------------------

% remove top header line
\renewcommand{\headrulewidth}{0pt} 

%remove botttom header line
\renewcommand{\footrulewidth}{0pt}	  	

%remove pagenum
\renewcommand{\thepage}{}	

%remove section num		
\renewcommand{\thesection}{}			


%----------------------------------------------------------------------------------------
%
% 	TIKZ GRAPHICS
%
%----------------------------------------------------------------------------------------

\newcounter{a}
\newcounter{b}
\newcounter{c}
\newcounter{barcount}

%----------------------------------------------------------------------------------------
% 	BAR CHART
%----------------------------------------------------------------------------------------

% draw a bar chart
% param 1: width
% param 2: height
% param 3: border color
% param 4: label text color
% param 5: label bg color
% param 6: cat 1 color
\newenvironment{barchart}[8]{
	
	\newcommand{\barwidth}{0.35}
	\newcommand{\barsep}{0.2}
	
	% param 1: overall percent
	% param 2: label
	% param 3: cat 1 percent
	% param 4: cat 2 percent
	% param 5: cat 3 percent
	\newcommand{\baritem}[5]{
		
		\pgfmathparse{##3+##4+##5}
		\let\perc\pgfmathresult
		
		\pgfmathparse{#2}
		\let\barsize\pgfmathresult
		
		\pgfmathparse{\barsize*##3/100}
		\let\barone\pgfmathresult
		
		\pgfmathparse{\barsize*##4/100}
		\let\bartwo\pgfmathresult
		
		\pgfmathparse{\barsize*##5/100}
		\let\barthree\pgfmathresult
		
		\pgfmathparse{(\barwidth*\thebarcount)+(\barsep*\thebarcount)}
		\let\barx\pgfmathresult
		
		\filldraw[fill=#6, draw=none] (0,-\barx) rectangle (\barone,-\barx-\barwidth);
		\filldraw[fill=#7, draw=none] (\barone, -\barx) rectangle (\barone+\bartwo,-\barx-\barwidth);
		\filldraw[fill=#8, draw=none] (\barone+\bartwo,-\barx ) rectangle (\barone+\bartwo+\barthree,-\barx-\barwidth);
		
		\node [label=180:\colorbox{#5}{\textcolor{#4}{##2}}] at (0,-\barx-0.175) {};
		\addtocounter{barcount}{1}
	}
	\begin{tikzpicture}
	\setcounter{barcount}{0}
	
}
{\end{tikzpicture}}

%----------------------------------------------------------------------------------------
% 	BUBBLE CHART
%----------------------------------------------------------------------------------------
\newcommand{\bubble}[5]{
	\definecolor{tmpcol}{RGB}{50,50,#5}
	% slice
	\filldraw[fill=black,draw=none] (#1,0.5) circle (#3);
	
	% outer label
	\node[label=\textcolor{black}{#4}] at (#1,0.7) {};
}

\newcommand{\bubbles}[2]{
	%reset counters
	\setcounter{a}{0}
	\setcounter{c}{150}
	\begin{tikzpicture}[scale=3]
	\foreach \p/\t in {#1} {
		\addtocounter{a}{1}
		\bubble{\thea/2}{\theb}{\p/25}{\t}{1\p0}
	}
	\end{tikzpicture}
}


%----------------------------------------------------------------------------------------
%	custom sections
%----------------------------------------------------------------------------------------

% create a coloured box with arrow and title as cv section headline
% param 1: section title
%
\newcommand{\cvsection}[1] {
	\textcolor{white}{\MakeUppercase{\textbf{#1}}}
}

\newcommand{\cvsect}[1]{
	\colorbox{black}{{\cvsection{#1}}}\\\\%
}

%----------------------------------------------------------------------------------------
% CUSTOM LOREM IPSUM
%----------------------------------------------------------------------------------------
\newcommand{\lorem}{Lorem ipsum dolor sit amet, consectetur adipiscing elit. Donec a diam lectus.}

%----------------------------------------------------------------------------------------
% ENTRY LIST
%----------------------------------------------------------------------------------------
\usepackage{tabularx}

\setlength{\tabcolsep}{0pt}
\newenvironment{entrylist}{%
	\begin{tabular*}{\textwidth}[t]{@{\extracolsep{\fill}}ll}
	}{%
	\end{tabular*}
}

\newcommand{\entry}[4]{%
	\parbox[t]{3.5cm}{%
		#1%
	}%
	&\parbox[t]{14cm}{%
		\textbf{#2}%
		\hfill%
		{\footnotesize \textbf{\textcolor{black}{#3}}}\\%
		#4%
	}\\\\}

\newcommand{\slashsep}{
	\hspace{2mm}/\hspace{2mm}
}

%----------------------------------------------------------------------------------------
%	DOCUMENT CONTENT
%----------------------------------------------------------------------------------------
\begin{document}
	
	%----------------------------------------------------------------------------------------
	%	TITLE HEADLINE
	%----------------------------------------------------------------------------------------
	\begin{minipage}[t]{0.45\textwidth}\hrule height 0pt width 0pt%
		\colorbox{black}{{\HUGE\textcolor{white}{\textbf{\MakeUppercase{Osain Abitia}}}}}%
		
		\vspace{1mm}\LARGE{Cloud Engineer}
	\end{minipage}%
	\begin{minipage}[t]{0.3\textwidth}\hrule height 0pt width 0pt%
		\small%
		\icontext{MapMarker}{12}{Mexico}\\
		\icontext{At}{12}{\href{mailto:osain.abitia@gmail.com}{osain.abitia@gmail.com}}\\	
	\end{minipage}%
	\begin{minipage}[t]{0.3\textwidth}\hrule height 0pt width 0pt%
		\small%
		\icontext{Github}{12}{\href{https://github.com/OsainAbitia}{github.com/OsainAbitia}}\\
		\icontext{Twitter}{12}{\href{https://twitter.com/@osainabitia}{twitter.com/@osainabitia}}\\
	\end{minipage}%
	
	% manage space by reducing font size
	\small%
	\vspace{0.5cm}
	
	%----------------------------------------------------------------------------------------
	%	SKILLS AND TECHNOLOGIES
	%----------------------------------------------------------------------------------------
	
	\cvsect{Who Am I?}%
	\begin{minipage}[t]{0.5\textwidth}%
		I'm a self-driven software engineer and certified Solutions Architect Associate. 
		Currently working as a Platform Engineer, I specialize in designing and implementing 
		cloud infrastructure using AWS. I take pride in creating tools that enhance the experience 
		for both teammates and users, always striving for excellence. I love to challenge myself
		continuously, learn from others and by my own.
	\end{minipage}%
	\hfill
	\begin{minipage}[t]{0.4\textwidth}\hrule height 0pt width 0pt%
		\vspace{-10pt}%
		\begin{barchart}{10}{5.5}{red}{white}{black}{black}{black}{black}
			\baritem{90}{AWS}{0}{0}{90}
			\baritem{80}{Python}{0}{0}{80}
			\baritem{80}{Terraform}{0}{0}{88}
			\baritem{90}{GitHub}{0}{0}{90}
			\baritem{40}{JavaScript}{0}{0}{40}
			\baritem{70}{Docker}{0}{0}{70}
		\end{barchart}
	\end{minipage}%
	\vspace{0.5cm}

	%----------------------------------------------------------------------------------------
	%	EXPERIENCE
	%----------------------------------------------------------------------------------------
	
	\cvsect{Experience}
	\begin{entrylist}
		\entry
		{Mar 2022 – Present}
		{Senior Platform Engineer 2}
		{Apli Jobs}
		{Driven by my passion for designing and implementing system architectures, 
		I pursued a career in platform engineering. I managed projects from initial 
		documentation and proposals to the creation of well-defined, scalable, and 
		highly available architectures. I established strong CI/CD pipelines that 
		automated processes such as testing and code quality assurance, significantly 
		enhancing the developer experience.\\
			\texttt{AWS}\slashsep\texttt{CI/CD}\slashsep\texttt{Python}\slashsep\texttt{Docker}\slashsep\texttt{Terraform}\slashsep\texttt{GCP}\slashsep\texttt{GitHub}}
		\entry
		{Jan 2022 – Mar 2022}
		{Cloud Engineer AWS Specialist}
		{Apli Jobs}
		{During my first years at Apli, I've led pivotal projects in infrastructure
		migration and optimization, successfully migrating critical applications to AWS
		within Docker containers for a comprehensive cloud modernization initiative. I
		played a role in ensuring ISO 27001 certification compliance and presented
		cloud-related evidence, emphasizing security protocols.
		\hfil\break
		Leveraging DevOps, I orchestrated CI/CD pipelines using AWS Elastic Beanstalk and
		ECS upgrades. I achieved substantial cost savings, operational efficiency, network
		security enhancements, and scalability. These initiatives aligned our tech framework
		with business imperatives, fostering sustained growth and a competitive edge in the
		dynamic realm of cloud technology.\\
			\texttt{AWS}\slashsep\texttt{CI/CD}\slashsep\texttt{Python}\slashsep\texttt{Docker}\slashsep\texttt{Terraform}\slashsep\texttt{GCP}\slashsep\texttt{GitHub}}
		\entry
		{Feb 2021 – Jan 2022}
		{Backend Developer + AWS Cloud Engineer}
		{Galaxy Technology}
		{As a Backend Engineer and Team Lead, I crafted REST APIs and oversaw database
		management using Node.js with Express and Sequelize. My role extended to hosting
		essential services within AWS, employing tools like Lambda, API Gateway,
		AppSync, DynamoDB, SNS, and SQS. Collaborating closely with customers, I addressed
		change requests and implemented new functionalities, enhancing our services to
		meet evolving needs.\\
			\texttt{Node.js}\slashsep\texttt{Express}\slashsep\texttt{Sequelize}\slashsep\texttt{AWS}\slashsep\texttt{GraphQL}}
	\end{entrylist}
	\\\\

	%----------------------------------------------------------------------------------------
	%	Education
	%----------------------------------------------------------------------------------------

	\cvsect{Education}
	\begin{entrylist}
		\entry
		{2017 – 2020}
		{Software Engineering}
		{Universidad Politécnica de Durango (UNIPOLI)}
		{Graduated with a 9.42 score. During this period, I had the chance
        to develop high-impact projects. As an outstanding example, the initial
        development of an AI for Lung Cancer detection was presented
        in Abu Dhabi, UAE.}
	\end{entrylist}
	\\\\
	\begin{minipage}[t]{0.3\textwidth}\hrule height 0pt width 0pt%
		\cvsect{Languages}
		\textbf{Spanish} - Native\\
		\textbf{English} - \href{https://cert.efset.org/hqaFfD}{C1}\\
	\end{minipage}%
	\hspace{0cm}
	\begin{minipage}[t]{0.7\textwidth}\hrule height 0pt width 0pt%
		\cvsect{Certificates}
		\href{https://www.credly.com/badges/dbb4510d-c80e-4bdf-bf9b-52e370013a9a}{AWS Certified Solutions Architect (Associate)}, 
		\href{https://www.credly.com/badges/4f129c39-edb2-410c-91aa-d8bf8617706f}{AWS Certified Cloud Practitioner}, 
		\href{https://www.credly.com/badges/1e0ac89f-4e38-4e91-93aa-3a62e7c33709}{AWS Well-Architected Proficient}, 
		\href{https://www.credly.com/badges/4f7bb440-1f63-4b79-a7b1-116f16da0656}{AWS Partner: Accreditation (Technical)}, 
		\href{https://www.credly.com/badges/c35e5c25-1938-4810-9489-1871644d7175}{AWS Partner: Accreditation (Business)}, 
	\end{minipage}%
	
\end{document}