%----------------------------------------------------------------------------------------
%	DOCUMENT DEFINITION
%----------------------------------------------------------------------------------------

% we use article class because we want to fully customize the page
\documentclass[10pt,A4]{article}	


%----------------------------------------------------------------------------------------
%	ENCODING
%----------------------------------------------------------------------------------------

%we use utf8 since we want to build from any machine
\usepackage[utf8]{inputenc}	

%----------------------------------------------------------------------------------------
%	LOGIC
%----------------------------------------------------------------------------------------

\usepackage{xifthen}
\usepackage{calc}
\usepackage[hidelinks]{hyperref}

%----------------------------------------------------------------------------------------
%	FONT
%----------------------------------------------------------------------------------------

% some tex-live fonts - choose your own

%\usepackage[defaultsans]{droidsans}
%\usepackage[default]{comfortaa}
%\usepackage{cmbright}
\usepackage[default]{raleway}
%\usepackage{fetamont}
%\usepackage[default]{gillius}
%\usepackage[light,math]{iwona}
%\usepackage[thin]{roboto} 

% set font default
\renewcommand*\familydefault{\sfdefault} 	
\usepackage[T1]{fontenc}

% more font size definitions
\usepackage{moresize}		

% font icons package
\usepackage{fontawesome}

%----------------------------------------------------------------------------------------
%	PAGE LAYOUT  DEFINITIONS
%----------------------------------------------------------------------------------------

%define page styles using geometry
\usepackage[a4paper]{geometry}		

% for example, change the margins to 2 inches all round
\geometry{top=2cm, bottom=2cm, left=2cm, right=2cm} 	

% use customized header
\usepackage{fancyhdr}				
\pagestyle{fancy}

%less space between header and content
\setlength{\headheight}{-5pt}		

% customize header entries
\lhead{}
\rhead{}
\chead{}

%indentation is zero
\setlength{\parindent}{0mm}

%----------------------------------------------------------------------------------------
%	GRAPHICS DEFINITIONS
%---------------------------------------------------------------------------------------- 

% for drawing graphics and charts
\usepackage{tikz}
\usetikzlibrary{shapes, backgrounds}

% use to vertically center content
% credits to: http://tex.stackexchange.com/questions/7219/how-to-vertically-center-two-images-next-to-each-other
\newcommand{\vcenteredinclude}[1]{\begingroup
	\setbox0=\hbox{\includegraphics{#1}}%
	\parbox{\wd0}{\box0}\endgroup}

% use to vertically center content
% credits to: http://tex.stackexchange.com/questions/7219/how-to-vertically-center-two-images-next-to-each-other
\newcommand*{\vcenteredhbox}[1]{\begingroup
	\setbox0=\hbox{#1}\parbox{\wd0}{\box0}\endgroup}

%----------------------------------------------------------------------------------------
%	ICON-SET EMBEDDING
%---------------------------------------------------------------------------------------- 

% at this point we simplify our icon-embedding by simply referring to a set of png images.
% if you find a good way of including svg without conflicting with other packages you can
% replace this part
\newcommand{\icon}[2]{\colorbox{black}{\makebox(#2, #2){\textcolor{white}{\large\csname fa#1\endcsname}}}}	%icon shortcut
\newcommand{\icontext}[3]{ 						%icon with text shortcut
	\vcenteredhbox{\icon{#1}{#2}}\hspace{0.2cm}\vcenteredhbox{\textcolor{black}{#3}}
}

%----------------------------------------------------------------------------------------
% 	HEADER
%----------------------------------------------------------------------------------------

% remove top header line
\renewcommand{\headrulewidth}{0pt} 

%remove botttom header line
\renewcommand{\footrulewidth}{0pt}	  	

%remove pagenum
\renewcommand{\thepage}{}	

%remove section num		
\renewcommand{\thesection}{}			


%----------------------------------------------------------------------------------------
%
% 	TIKZ GRAPHICS
%
%----------------------------------------------------------------------------------------

\newcounter{a}
\newcounter{b}
\newcounter{c}
\newcounter{barcount}

%----------------------------------------------------------------------------------------
% 	BAR CHART
%----------------------------------------------------------------------------------------

% draw a bar chart
% param 1: width
% param 2: height
% param 3: border color
% param 4: label text color
% param 5: label bg color
% param 6: cat 1 color
\newenvironment{barchart}[8]{
	
	\newcommand{\barwidth}{0.35}
	\newcommand{\barsep}{0.2}
	
	% param 1: overall percent
	% param 2: label
	% param 3: cat 1 percent
	% param 4: cat 2 percent
	% param 5: cat 3 percent
	\newcommand{\baritem}[5]{
		
		\pgfmathparse{##3+##4+##5}
		\let\perc\pgfmathresult
		
		\pgfmathparse{#2}
		\let\barsize\pgfmathresult
		
		\pgfmathparse{\barsize*##3/100}
		\let\barone\pgfmathresult
		
		\pgfmathparse{\barsize*##4/100}
		\let\bartwo\pgfmathresult
		
		\pgfmathparse{\barsize*##5/100}
		\let\barthree\pgfmathresult
		
		\pgfmathparse{(\barwidth*\thebarcount)+(\barsep*\thebarcount)}
		\let\barx\pgfmathresult
		
		\filldraw[fill=#6, draw=none] (0,-\barx) rectangle (\barone,-\barx-\barwidth);
		\filldraw[fill=#7, draw=none] (\barone, -\barx) rectangle (\barone+\bartwo,-\barx-\barwidth);
		\filldraw[fill=#8, draw=none] (\barone+\bartwo,-\barx ) rectangle (\barone+\bartwo+\barthree,-\barx-\barwidth);
		
		\node [label=180:\colorbox{#5}{\textcolor{#4}{##2}}] at (0,-\barx-0.175) {};
		\addtocounter{barcount}{1}
	}
	\begin{tikzpicture}
	\setcounter{barcount}{0}
	
}
{\end{tikzpicture}}

%----------------------------------------------------------------------------------------
% 	BUBBLE CHART
%----------------------------------------------------------------------------------------
\newcommand{\bubble}[5]{
	\definecolor{tmpcol}{RGB}{50,50,#5}
	% slice
	\filldraw[fill=black,draw=none] (#1,0.5) circle (#3);
	
	% outer label
	\node[label=\textcolor{black}{#4}] at (#1,0.7) {};
}

\newcommand{\bubbles}[2]{
	%reset counters
	\setcounter{a}{0}
	\setcounter{c}{150}
	\begin{tikzpicture}[scale=3]
	\foreach \p/\t in {#1} {
		\addtocounter{a}{1}
		\bubble{\thea/2}{\theb}{\p/25}{\t}{1\p0}
	}
	\end{tikzpicture}
}


%----------------------------------------------------------------------------------------
%	custom sections
%----------------------------------------------------------------------------------------

% create a coloured box with arrow and title as cv section headline
% param 1: section title
%
\newcommand{\cvsection}[1] {
	\textcolor{white}{\MakeUppercase{\textbf{#1}}}
}

\newcommand{\cvsect}[1]{
	\colorbox{black}{{\cvsection{#1}}}\\\\%
}

%----------------------------------------------------------------------------------------
% CUSTOM LOREM IPSUM
%----------------------------------------------------------------------------------------
\newcommand{\lorem}{Lorem ipsum dolor sit amet, consectetur adipiscing elit. Donec a diam lectus.}

%----------------------------------------------------------------------------------------
% ENTRY LIST
%----------------------------------------------------------------------------------------
\usepackage{tabularx}

\setlength{\tabcolsep}{0pt}
\newenvironment{entrylist}{%
	\begin{tabular*}{\textwidth}[t]{@{\extracolsep{\fill}}ll}
	}{%
	\end{tabular*}
}

\newcommand{\entry}[4]{%
	\parbox[t]{3.5cm}{%
		#1%
	}%
	&\parbox[t]{14cm}{%
		\textbf{#2}%
		\hfill%
		{\footnotesize \textbf{\textcolor{black}{#3}}}\\%
		#4%
	}\\\\}

\newcommand{\slashsep}{
	\hspace{2mm}/\hspace{2mm}
}

%----------------------------------------------------------------------------------------
%	DOCUMENT CONTENT
%----------------------------------------------------------------------------------------
\begin{document}
	
	%----------------------------------------------------------------------------------------
	%	TITLE HEADLINE
	%----------------------------------------------------------------------------------------
	\begin{minipage}[t]{0.45\textwidth}\hrule height 0pt width 0pt%
		\colorbox{black}{{\HUGE\textcolor{white}{\textbf{\MakeUppercase{Osain Abitia}}}}}%
		
		\vspace{1mm}\LARGE{Cloud Engineer}
	\end{minipage}%
	\begin{minipage}[t]{0.3\textwidth}\hrule height 0pt width 0pt%
		\small%
		\icontext{MapMarker}{12}{Mexico}\\
		\icontext{At}{12}{\href{mailto:osain.abitia@gmail.com}{osain.abitia@gmail.com}}\\	
	\end{minipage}%
	\begin{minipage}[t]{0.3\textwidth}\hrule height 0pt width 0pt%
		\small%
		\icontext{Github}{12}{\href{https://github.com/OsainAbitia}{github.com/OsainAbitia}}\\
		\icontext{Twitter}{12}{\href{https://twitter.com/@osainabitia}{twitter.com/@osainabitia}}\\
	\end{minipage}%
	
	% manage space by reducing font size
	\small%
	\vspace{0.5cm}
	
	%----------------------------------------------------------------------------------------
	%	SKILLS AND TECHNOLOGIES
	%----------------------------------------------------------------------------------------
	
	\cvsect{Who Am I?}%
	\begin{minipage}[t]{0.5\textwidth}%
		I'm a software engineer who enjoys continuous self-learning
		and collaboration with peers. Always pursuing growth
		opportunities, I'm currently a Cloud Engineer certified as
		Solutions Architect Associate. In this role, I fuel my passion
		for designing and implementing cloud infrastructure, leveraging
		potent tools like Docker containers and CI/CD in GitHub actions.
		I utilize cloud-native services like CodePipeline, CodeBuild, and
		CodeDeploy, prioritizing Infrastructure as Code. Crafting useful
		tools for teammates and users brings me joy, as I strive for the
		highest standards in my work
	\end{minipage}%
	\hfill
	\begin{minipage}[t]{0.4\textwidth}\hrule height 0pt width 0pt%
		\vspace{-10pt}%
		\begin{barchart}{10}{5.5}{red}{white}{black}{black}{black}{black}
			\baritem{50}{AWS}{0}{0}{90}
			\baritem{80}{Python}{0}{0}{80}
			\baritem{80}{Terraform}{0}{0}{88}
			\baritem{80}{Git}{0}{0}{70}
			\baritem{40}{JavaScript}{0}{0}{40}
			\baritem{50}{Docker}{0}{0}{60}
		\end{barchart}
	\end{minipage}%
	\vspace{0.5cm}

	%----------------------------------------------------------------------------------------
	%	EXPERIENCE
	%----------------------------------------------------------------------------------------
	
	\cvsect{Experience}
	\begin{entrylist}
		\entry
		{Jan 2022 – Present}
		{Cloud Engineer}
		{Apli Jobs.}
		{Cloud Architect with AWS services, focusing on designing and creating
        infrastructure with IaC tools such as Terraform. Enhancing and building CICD
        pipelines with the help of GitHub actions/AWS native tools.
        Vital participant in the achievement of getting the ISO 27001 certification at the
        company, supporting cloud control evidence and its application.\\
			\texttt{AWS}\slashsep\texttt{CICD}\slashsep\texttt{Python}\slashsep\texttt{Docker}\slashsep\texttt{Terraform}\slashsep\texttt{GCP}}
		\entry
		{Feb 2021 – Jan 2022}
		{AWS Cloud Engineer + Backend Developer}
		{Galaxy Technology}
		{Backend engineer and team lead on developing REST APIs and Database management
        with Node.js with Express and Sequelize, respectively. Hosting critical services within
        AWS and using tools such as Lambda, API Gateway, AppSync, Dynamo, SNS, and SQS.\\
			\texttt{Node.js}\slashsep\texttt{Express}\slashsep\texttt{Sequelize}\slashsep\texttt{AWS}\slashsep\texttt{GraphQL}}
		\entry
		{April – Oct 2020}
		{Junior React Native Developer}
		{Cubo Rojo}
		{Full-stack developer for a React Native app powered by Firebase with a non-relational
        database and as a user authentication handler. The mentioned application helped the company 
        to automate company users' requests and processes for computers configuration, reducing the 
		company response time from ~30 minutes to real time interaction.\\
			\texttt{React Native}\slashsep\texttt{Node.js}\slashsep\texttt{Python}\slashsep\texttt{Ubuntu Server}\slashsep\texttt{Firebase}}
	\end{entrylist}
	\\\\

	%----------------------------------------------------------------------------------------
	%	Education
	%----------------------------------------------------------------------------------------

	\cvsect{Education}
	\begin{entrylist}
		\entry
		{2017 – 2020}
		{Software Engineering}
		{Universidad Politécnica de Durango (UNIPOLI)}
		{\lorem\lorem\lorem}
	\end{entrylist}
	\\\\
	\begin{minipage}[t]{0.3\textwidth}\hrule height 0pt width 0pt%
		\cvsect{Languages}
		\textbf{Czech} - native\\
		\textbf{English} - C2\\
		\textbf{Polish} - B1
	\end{minipage}%
	\hspace{0cm}
	\begin{minipage}[t]{0.3\textwidth}\hrule height 0pt width 0pt%
		\cvsect{Hobbies}
		I love... \lorem
	\end{minipage}%
	\hspace{2cm}
	\begin{minipage}[t]{0.3\textwidth}\hrule height 0pt width 0pt%
		\cvsect{Non profit}
		I help... \lorem
	\end{minipage}%
	% LANGUAGES
	
\end{document}